% File: slides.tex
% Location: ./course_instances/Sp26/lectures/01_intro/slides.tex

\documentclass[10pt]{beamer}

% =======================================================================
% 1. THEMES AND APPEARANCE
% =======================================================================

% Theme: Boadilla is a clean, standard theme with a visible sidebar.
\usetheme{Boadilla}

% Color Theme: Seahorse is professional (blue/teal) and high-contrast.
\usecolortheme{seahorse}

% Font and Encoding
\usepackage[utf8]{inputenc}
\usepackage[T1]{fontenc}

% Graphics and Math Packages (essential for a technical course)
\usepackage{graphicx}
\usepackage{amsmath}
\usepackage{amssymb}
\usepackage{hyperref}

% Code Highlighting (Crucial for showing Fortran snippets)
\usepackage{listings}
\lstset{
    language=Fortran,
    basicstyle=\ttfamily\footnotesize, % Smaller, fixed-width font
    numbers=left,
    numberstyle=\tiny,
    stepnumber=1,
    numbersep=5pt,
    backgroundcolor=\color[gray]{0.95}, % Light grey background for code
    showspaces=false,
    showstringspaces=false,
    showtabs=false,
    frame=single,
    rulecolor=\color{gray},
    captionpos=b,
    breaklines=true,
    breakatwhitespace=true,
    tabsize=4,
    keywordstyle=\color{blue},
    commentstyle=\color{green!50!black},
    stringstyle=\color{red}
}


% =======================================================================
% 2. DOCUMENT METADATA (Change for each lecture)
% =======================================================================

\title[Lecture 1: Intro]{Introduction to Next Generation Data Centers}
%\subtitle{Data Center Architecture and Energy Challenges}
\author[Thomas Schutzius]{Prof. Thomas Schutzius}
\institute{Department of Mechanical Engineering\\
    \textit{University of California, Berkeley}}
\date{Spring 2026 -- Lecture 1}


% =======================================================================
% 3. BEGIN DOCUMENT
% =======================================================================

\begin{document}

% --- Title Page ---
\begin{frame}
    \titlepage
\end{frame}

% --- Table of Contents (Outline) ---
\begin{frame}{Lecture Outline}
    \tableofcontents
\end{frame}


% =======================================================================
% 4. CONTENT SECTIONS
% =======================================================================

\section{The Data Center Energy Landscape}

\begin{frame}{Current Energy Challenges}
    \begin{itemize}
        \item Global data center energy consumption is rapidly growing.
        \item **PUE (Power Usage Effectiveness)** is the key metric:
        $$
        \text{PUE} = \frac{\text{Total Facility Energy}}{\text{IT Equipment Energy}}
        $$
        \pause % Reveals the next item on a click
        \item Cooling load dominates non-IT energy (fans, chillers, pumps).
        \item Goal: Drive PUE closer to $1.0$ through efficient design.
    \end{itemize}
\end{frame}

\section{Computational Models}

\subsection{Thermal Modeling}

\begin{frame}{Modeling Heat Transfer (Example)}
    \begin{block}{Heat Dissipation in a Server Rack}
        The transient temperature $T$ of a component is modeled by:
        $$
        m c_p \frac{dT}{dt} = \dot{Q}_{\text{gen}} - \dot{Q}_{\text{conv}}
        $$
        \begin{itemize}
            \item $m$: Mass of the component (kg)
            \item $c_p$: Specific heat capacity (J/kg $\cdot$ K)
            \end{itemize}
    \end{block}
\end{frame}

\subsection{Fortran Code Example}

\begin{frame}{Example: Basic Fortran Simulation}
    \frametitle{Running a Simple Energy Loop}

    \begin{itemize}
        \item We use Fortran for high-speed, parallel thermal simulations.
    \end{itemize}

% \begin{lstlisting}[caption=Initialization Loop (basic\_loop.f90)]
%program basic_loop
 % implicit none
%  integer :: i
%  real :: total_energy = 0.0, power_draw = 1500.0 ! Watts

%  do i = 1, 10
%    total_energy = total_energy + power_draw
%  end do
  
%  print *, "Total Energy over 10 seconds: ", total_energy, " Joules"
%end program basic_loop
%	\end{lstlisting}
    
\end{frame}

% --- Conclusion Slide ---
\section{Conclusion}

\begin{frame}{Summary and Next Steps}
    \frametitle{Wrap-up}
    \begin{itemize}
        \item Next Gen DCs require co-design of IT and Cooling systems.
        \item **Key Takeaway**: Energy efficiency is a system-level problem.
    \end{itemize}
    \vfill % Add vertical space
    \centering
    \alert{Reading for Next Week: Chapter 2 on Advanced Cooling Techniques}
\end{frame}

\end{document}